\documentclass[../main/main.tex]{subfiles}

\begin{document}

\section*{EXAMPLE CODE}
These are example code for future use and manipulation.

%-------------------------------------------------------------------------------------------------------------%
% LISTS
%-------------------------------------------------------------------------------------------------------------%
\subsection*{Making list in list in list...}
\textbf{Itemized lists}
\begin{itemize}
\item item 1
\item item 2
\item item 3
	\begin{itemize}
	\item subitem 1
	\item subitem 2
		\begin{itemize}
		\item subsubitem 1
		\end{itemize}
	\end{itemize}
\end{itemize}

\noindent \textbf{Enumerated lists}
\begin{enumerate}
\item item 1
\item item 2
\item item 3
	\begin{enumerate}
	\item subitem 1
	\item subitem 2
		\begin{enumerate}
		\item subsubitem 1
		\end{enumerate}
	\end{enumerate}
\end{enumerate}

\noindent \textbf{Changing item headings. More useful tips can be found here: \url{https://www.latex-tutorial.com/tutorials/lists/}}
\begin{enumerate}[label=\Alph*.]
\item this is item a
\item another item
\end{enumerate}

\noindent \textbf{Changing numbering}
\begin{enumerate}
\item this is item a
\setcounter{enumi}{5}
\item another item
\end{enumerate}

\noindent \textbf{Customizing one item at a time}
\begin{itemize}
\item  Default item label for entry one
\item  Default item label for entry two
\item[$\square$]  Custom item label for entry three
\end{itemize}

%-------------------------------------------------------------------------------------------------------------%
% PARTITION PAGE
%-------------------------------------------------------------------------------------------------------------%
\subsection*{Manipulating page organization}
\subsubsection*{Spacing}
First line\\ \vspace{\baselineskip}
Third line

%-------------------------------------------------------------------------------------------------------------%
% FIGURES
%-------------------------------------------------------------------------------------------------------------%
\subsection*{Figure fundamentals}

%-------------------------------------------------------------------------------------------------------------%
% TABLES
%-------------------------------------------------------------------------------------------------------------%
\subsection*{Table fundamentals}

\begin{table}[]
\begin{tabular}{llll}
\toprule
Name & AA     & fwd                & rev                \\
\midrule
mut1 & GCCGCC & GGTTGTTGTGGATGCTGT & ACAGCATCCACAACAACC \\
mut2 & CCGCCG & TGCTGTGGGTGCTGTGGC & GCCACAGCACCCACAGCA \\
mut3 & CGGCGG & TGTGGCGGCTGTGGCGGC & GCCGCCACAGCCGCCACA \\
mut4 & GCGGCG & GGATGCGGAGGCTGCGGA & TCCGCAGCCTCCGCATCC \\
mut5 & GGCGGC & GGTGGTTGCGGTGGGTGC & GCACCCACCGCAACCACC \\
\bottomrule
\end{tabular}
\end{table}

%-------------------------------------------------------------------------------------------------------------%
% EQUATIONS
%-------------------------------------------------------------------------------------------------------------%
\subsection*{Equation fundamentals}


$$
\text{GEORGE} = \iiint_\text{awespme}^\text{COOL}
$$
\end{document}
