%---------------------------------------------------------------------------%
%								   Preamble									%
%---------------------------------------------------------------------------%
\documentclass[12pt, lettersize]{article}
\usepackage{preamble}
\graphicspath{{../images/}}
\captionsetup{font=footnotesize}
% note that graphics path should be encased in {}, ends with / to define directory, and finally separated by , without any lead or lag spaces
%---------------------------------------------------------------------------%
%								Begin Document								%
%---------------------------------------------------------------------------%
\begin{document}

%------------------
% Table of Contents
%------------------
\begin{center}
	%-------------% 
	% Institution %
	%-------------%
	{\LARGE\bf Massachusetts Institute of Technology \\
	\vspace{0.25\baselineskip}
	Biological Engineering Department}	
	\vspace{\baselineskip}
	
	%----------% 
	% Proposal %
	%----------%		
	{\Large\bf Thesis Proposal (OR Defense) \\
	\vspace{0.25\baselineskip}
	Doctor of Philosophy}		
	\vspace{4\baselineskip}

	%--------------% 
	% Thesis Title %
	%--------------%	
	% {\Large\bf\underline{Title}:} \\	
	\vspace{2\baselineskip}	
	{\LARGE\bf FIRST LINE TITLE \\
			   \vspace{0.25\baselineskip}
			   SECOND LINE TITLE}
	\vspace{3\baselineskip}

	%--------------------% 
	% Date of Submission %
	%--------------------%			
	Date of Submission: \\	
	\vspace{0.5\baselineskip}
	\today
	% September 13\textsuperscript{th}, 2016
		
	\vspace{8\baselineskip}	
		
	\begin{tabular}{rlc}
		%------------------% 
		% Author Signature %
		%------------------%		
		{\small \sc Submitted by:}
	        	                    & YOUR NAME  & \\
		\\ %space	
		{\small \sc Signed:} & \cline{1-1} \\ 
			
		%----------------------% 
		% Supervisor Signature %
		%----------------------%	
		{\small \sc Supervisor:}
	        	                    & SUPERVISOR NAME  & \\
	            	                & Professor of DEPARTMENT \& & \\
	            	                & SECOND DEPARTMENT TITLE & \\
		\\ %space
		{\small \sc Signed:} & \cline{1-1} \\
	                         
		%---------------------------% 
		% Academic Office Signature %
		%---------------------------%	                            			    
		{\small \sc Academic Office:}
								& ACADEMIC COORDINATOR  & \\
	                            & Academic Administrator & \\
	                            & YOUR DEPARTMENT & \\
		\\ %space	                            
		{\small \sc Signed:} & \cline{1-1} \\	    	
	\end{tabular}	
\end{center}


\newpage

%------------------
% Table of Contents
%------------------
% \thispagestyle{empty}

\tableofcontents 
%\listoffigures 
%\listoftables
 
\newpage
 
\pagenumbering{arabic}

%-------------------------------------------------------------------------------------------------------------%
% Abstract
%-------------------------------------------------------------------------------------------------------------%
\begin{abstract}
Abstract should be no more than \textbf{300 words in 1 page}.
\vspace{\baselineskip}

\noindent State the significance of the proposed research. Include long-term objectives and specific aims. Describe concisely the research design and methods for achieving these objectives. Highlight the specific
hypotheses to be tested, goals to be reached, or technology to be developed, which are intended to be
your original contributions. Avoid summaries of past accomplishments.

\end{abstract}

%-------------------------------------------------------------------------------------------------------------%
% Objective
%-------------------------------------------------------------------------------------------------------------%
\section{OBJECTIVE}
Overall Objective \& Specific Aims should be \textbf{1 page maximum}.

\subsection{OBJECTIVE SUBSECTION}
\noindent Articulate the overall objective of your thesis project, and outline a set of specific aims by which your work
is intended to accomplish this objective. Be sure to clearly state the hypotheses to be tested, goals to be
reached, or technology to be developed.

%-------------------------------------------------------------------------------------------------------------%
% Background
%-------------------------------------------------------------------------------------------------------------%
\section{BACKGROUND}
Background \& Significance section should be \textbf{3-5 pages}.

\subsection{BACKGROUND SUBSECTION}
\noindent Sketch the background leading to the present research, critically evaluate existing knowledge, and
specifically identify the gaps that your research is intended to fill. State concisely the importance of the
research described in this proposal by relating the specific aims to the broad, long-term objectives.

%-------------------------------------------------------------------------------------------------------------%
% Research Design & Methods
%-------------------------------------------------------------------------------------------------------------%
\section{METHODS} \label{section:methods}
Research Design \& Methods section should be \textbf{3-5 pages}.

\subsection{METHODS SUBSECTION}
\noindent Along with the Objective \& Aims section, this is the most important part of the proposal. The majority of
your time should be spent making this part of your proposal strong, direct, and completely clear. Describe
the research design and the procedures to be used to accomplish the specific aims of the project; it is
generally most effective to do this according to the same outline as in the Objective \& Aims section. Include
how the data will be collected, analyzed, and interpreted. Describe any new methodology and its advantage
over existing methodologies. Discuss the potential difficulties and limitations of the proposed procedures
and alternative approaches to achieve the aims. As part of this section, provide a tentative timetable for the
project. Point out any procedures, situations or materials that may be hazardous and the precautions to be
exercised.

%-------------------------------------------------------------------------------------------------------------%
% Preliminary Results
%-------------------------------------------------------------------------------------------------------------%
\section{RESULTS}
Preliminary Studies section should be \textbf{3-4 pages}. \\\\
This section may be alternatively be located before the Research Design \& Methods Section \ref{section:methods}.

\subsection{RESULTS SUBSECTION}
\noindent Use this section to provide an account of your preliminary studies that are pertinent to your research project
and that support your specific aims. Note: it is not necessary to have obtained a substantial amount of
preliminary data in order to submit or defend the proposal, although it will be expected that you have begun
to undertake some of the key methods to assess their feasibility.

%-------------------------------------------------------------------------------------------------------------%
% Bibliography
%-------------------------------------------------------------------------------------------------------------%
\newpage
% \bibliographystyle{unsrt} % unsorted bibliography
% \bibliography{./bib/..}

% manual bib insertion
\begin{thebibliography}{99} % means 1 digit of entries (<10)

\bibitem{FSR}
THIS IS A BIBITEM

\end{thebibliography}

%-------------------------------------------------------------------------------------------------------------%
% Appendix
%-------------------------------------------------------------------------------------------------------------%
\newpage
\appendix
\section{APPENDIX}

\subsection{SUB APPENDIX}

%---------------------------------------------------------------------------%
%								 End Document								%
%---------------------------------------------------------------------------%
\end{document}
